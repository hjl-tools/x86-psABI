\chapter{About this Document\label{intro}}

This document is a supplement to the existing \xARCH System V
Application Binary Interface (ABI) document available at
\url{http://www.sco.com/developers/devspecs/abi386-4.pdf},
which describes the ABI for processors compatible with the
\xARCH Architecture.

This document describes the conventions and constraints on the
implementation of these new features for interoperability between
various tools.

\section{Scope}

This document describes the conventions on the new C/C++ language types
(including alignment and parameter passing conventions), the relocation
symbols in the object binary, and the exception handling mechanism for
\xARCH architecture.  Some of this work has been discussed before
\url{http://groups.google.com/group/ia32-abi} or
\url{http://www.akkadia.org/drepper/tls.pdf}. The C++ object model that
is expected to be followed is described in
\url{http://mentorembedded.github.io/cxx-abi/}.  In particular,
this document specifies the information that compilers have to generate
and the library routines that do the frame unwinding for exception
handling.

\section{Related Information}

Links to useful documents:
\begin{itemize}
 \item System V Application Binary Interface, Intel386{\texttrademark} Architecture
       Processor Supplement Fourth Edition:
       \url{http://www.sco.com/developers/devspecs/abi386-4.pdf}
 \item System V Application Binary Interface, AMD64 Architecture Processor
       Supplement, Draft Version 0.99.6:
       \url{http://www.x86-64.org/documentation/abi.pdf}
 \item Discussion of Intel processor extensions:
       \url{http://groups.google.com/group/ia32-abi}
 \item ELF Handling of Thread-Local Storage:
       \url{http://www.akkadia.org/drepper/tls.pdf}
 \item Thread-Local Storage Descriptors for IA32 and AMD64/EM64T:
       \url{http://www.fsfla.org/~lxoliva/writeups/TLS/RFC-TLSDESC-x86.txt}
 \item Itanium C++ ABI, Revised March 20, 2001:
       \url{http://mentorembedded.github.io/cxx-abi/}
\end{itemize}

%%% Local Variables:
%%% mode: latex
%%% TeX-master: "abi"
%%% End:
