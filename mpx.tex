%%% vim:ai:tw=72:
\chapter{Intel MPX Extension}

Intel MPX (Memory Protection Extensions) provides 4 64-bit wide bound
registers (\reg{bnd0} - \reg{bnd3}).  For purpose of function return,
the lower 32 bits of \reg{bnd0} specify lower bound of function return,
and the upper 32 bits specify upper bound of function return.  The upper
bound is represented in one's complement form.

\section{Parameter Passing and Returning of Values}
\label{mpx-calling-conventions}

\subsection{Bounds Passing}
\label{bounds_passing}
Intel MPX provides ISA extensions that allow passing bounds for a pointer
argument that specify memory area that may be legally accessed by
dereferencing the pointer.  This paragraph desribes how the bounds are
passed to the callee.

Several functions used in the description below are defined as follows:
\begin{description}
\item[BOUND_MAP_STORE(bnd, addr, ptr)] This function executes Intel MPX \code{bndstx}
  instruction.  \code{ptr} argument is used to initialize index field of the memory
  operand of the \code{bndstx} instruction, \code{addr} is encoded in base and/or
  displacement fields of the memory operand, \code{bnd} is encoded in the register
  operand.
\item[BOUND_MAP_LOAD(addr, ptr)] This function executes Intel MPX \code{bndldx}
  instruction. \code{ptr} argument is used to initialize index field of the memory
  operand of the \code{bndldx} instruction, \code{addr} is encoded in base and/or
  displacement fields of the memory operand.
\end{description}

The bounds associated with each pointer contained in the fourbyte are
passed in a CPU defined manner by executing
\code{BOUND_MAP_STORE(bnd, addr, ptr)} function, where \code{bnd} is the
current bounds of the pointer argument, \code{addr} is the address of the
pointer argument's stack location, \code{ptr} is the actual value of the
pointer argument.  If the fourbyte may contain parts of partially
overlapping pointers, then bounds associated with the pointers are ignored
and special bounds that allow accessing all memory are passed for such
pointers.  The callee fetches the passed bounds using
\code{BOUND_MAP_LOAD(addr, ptr)}, where \code{addr} is the same address
passed to the corresponding \code{BOUND_MAP_STORE} in the caller, and
\code{ptr} is the actual value of the pointer parameter fetched by the
callee from a stack location.

When passing arguments with bounds to functions, function prototypes
must be provided.  Otherwise, the run-time behavior is undefined.

\subsection{Returning of Bounds}
The returning of bounds is done according to the following algorithm:
\begin{enumerate}
\item When the value is returned in memory, on return \reg{bnd0} must
  contain bounds of the ``hidden'' first argument that has been passed
  in by the caller.

\item When a pointer value is returned, on return \reg{bnd0} must
  contain bounds of the pointer value.
\end{enumerate}

%%% Local Variables:
%%% mode: latex
%%% TeX-master: "abi"
%%% End:
